% CHAPTER 4 LESSON 3
\clearpage
\section{Synthesis: Overlap-add technique}
\label{Synthesis: Overlap-add technique}

The typical way that we work with signals in the frequency domain is that we first analyze it. Then we do some modifications to it, lets say we have a noisy signal, lets say there was some construction work noise and we want to get rid of it, so we want to apply noise reduction, or we want to reduce echos if the are undesired, or we could stretch the signal in time or do pitch shifting and once we are done with all of these modification, we have to synthesisze the time domian signal again because we can only listen, our loud speaker is only capable of reproducing time domain signals. Thats the reason why we have to get back to time domain. So what we do is ...

We first take the segement, the overlapping segments, we apply a window again so its a lot of repetition and then compute the DFT. So lets say this is the analysis window here, we do some magic in the modification step and then we want to go back so we apply the inverse DFT. SO if there is no modificaltion, we have w(n)times xl(n). So this small part of the whole signal x is called xl.  So we bring it to the DFT domain then we directly bring it back to the time domain, so we have w(n)*xl(n). SO we basically have these signals again. SO in the next step we have to get back to our time domain signal because right now we just have these separate segment. SO now we align them in the same way that we took them out of the singal, so overlapping lets say by 50 percent and then we add them up.  And then come out with something that basically looks like this.  SO they should ideally always add up to one, these windows.  Becasue in case that they would not do it, we would not have what we call perfect reconstruction. So what we want to have is apply the STFT then the inverse STFT and then we get exactly the same signal back gain. And this is only possible if this addition here, this signal sums up to x(n) again up to some offset, lets call it delta. So the most, the onlydifference that should be between that one and the original signal should be one offset here, one time domain offset. Not more than that.  That is something that we are definitely aiiming for. 

ANd for that the windows and the overlap need to fulfill a specific relation. Mathematically we can state it like this :
So yl(n) is the windowed segemtn, x(n) is the segemnt wihtout a window. And w(n) is our analysis window. And we bring it to the DFT domain and back to the time domain .  ANd then we apply a synthesis window.  Synthesis window I did not introduce yet, I will talk about that shortly, but lets for now assume that we need a synthesis window that is after the DFT and the IDFT. We aplly another windo which is called v(n) soin the end. We have something that is:

So mathematically, we can say that this summation here should always be one, that is what I tried to draw there and here are some examples of that.  Here we don not apply a synthesis window and we just take v(n) to be one(rectangular), so then this formula on the bottom right reduces to that one so that the analysis window should add up tot a constant value, so here we use a triangular analysis window, a hamming window and and Hann window and you can see that if you add them all up one after the other in thhis overlap fashion the nyou end up with a constant value here.  SO this is just like boundary issues, in the very beginning and in the very end but in between tehre should be always one. But that is not guarateed all of the time.  SO lets say here, we take a Hann window for example, in this case it adds up to 2, this is an overlap of 75percent it adds up exactly to 2, thats nice.  But you could also have it that this is not the case like if you had a strange overlap of 30 percent or something.  Then this would not add up to one any more but would add up to something like this. So this would result in modulations in the reconstructed signal that are not desired.  SO its not the same window that we put into there anymore.  So for lets say for a Hann and Hamming window toachieve perfect reconstruction we need overlaps of 50 percent, 75 percent, 87 percent.  The shift needs to always be at 2 to the power of -somehting to be accurate.  You always have to take care that your choice of window length, shift and the form of window are such that it allows for perfect reconstrucution. 

So the most frequently used one is 50 perccent, because if you use 75 percent you have double the numberof segement s that you have double the computational load that you need to process it and in our area, lets say we goto hearing aids and computtauonal load is extremely important becasue we only have a short battery life and for hearing aid users to accept a hearing aid, it needs to run a t least a week without changing the battery. so you really need to take care about power consumption.

So now we come to the synthesis part and go more into detail becasue now we assum that we have done something to the signal.  Upto now we have only go in and out, but lets say we did do some modifications to it. So thats a segment, and then we apply a Hann window to it, then we bring it to the frequency domain then we do some modification, lets say we multiply it with some function Gomega()and multiply it with the windowed function Xw(omega) and go back to time domain. In the case that G is one, then we would get the same signal, everything is fine again, no problems at all.  And but in the case that we apply some G here that is a multiplecation in the frequency domain, so we get a conviltuion in the time domain. SO lets say this one here, if you bring it to the time domain is g(n), and output is something that is g(n)convolved with x(n)w(n).  Its not the same signal anymore, but its also what we are aiming at. Lets say the g looks something liek this and we convolve this with the original signal here. And what would happen is herer(beginning ) the values would be different from zero and the output would look something liek this, and you cannot see the window anymore. Ans this could be problematic, becasue here you have these boundaries, and boundaries are always bad, so if you start overlapping and adding them again, then you have here some parts where it is not zero and the other part is the same and when they add up, you might end up with some artifacts that are audible. and that you dont want to have.  And this is the reason that we often apply a synthesis window.SO after all of this processing, we again apply a window that looks something like a Hann window, to reduce these boundary issues here. And with that, we reduce the audible artifacts.  Thats one solution to the problem, another solution could be that we use an analyis window, but apply zeropadding.  That could also help. 

Lets say g has a length Lg, and x(n)(w) has a length of N.  SO if you convolve two signals of these length, the length is L+N-1. And this is longer than just N. So there could be something like this there, but since it is not therre, we fight with circular convolution, the stuff thats in a linear convolution would show up here, shows up here again(beginning) and this is what is coming up in this area here and  this is what produces these artifacts. So lets now again ask the question, why would zero padding help to avoid that. So basically the additional length of the signals is all zeros but that really depends on g, that can be really long, 

So these two ways exist, so you either append zeros here or you apply a synthesis window at the very end, where the synthesis window is also just an approximation solutionto it becasue we still see some of the errors in there, but they are reduced.  The synthesis window has the benefit that it introduces minimal computational costs. WHile here, adding zeros to it introduces additional costs in computing the STFT, then there are more frequency bands in the STFT and there is more computational overhead. 

And how can we choose the synthesis window. One typical choice is the square root Hann window. So we said that for a 50percent overlap, the Hann window produces perfect reconstruction of the original signal if we don not apply any modifications here. However, if we apply a synthesis window here, then by using the Hann window afterwards again a different window, would not be capable of producing perfect reconstruction anymore.  SO what we need is not that w(n) allows for perfect reconstruction, but w(n)v(n) so the multiplecation of the analysis window and the synthesis window that is basically, applied.  These two need to allow for perfect reconstruction and we said that if v(n) is one, then we can use the Hann window for w(n), but another choice is that we use the square root of Hann here and the square root of Hann there and  that allows for perfect reconstruction.

