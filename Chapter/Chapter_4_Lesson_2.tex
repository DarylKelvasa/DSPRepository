% CHAPTER 4 LESSON 2
\clearpage
\section{Spectral Envelope}
\label{Spectral Envelope}


Envelope

So here we have one speech sentence.  And we pick put three representative segments. WHich is an f, an i, and an u. and we have a look at the spectrum, the DFT of the short signal segments. And you can see that for the f there are rather high frequency spectral components, but for the i and the u you see the formant structure, but also the fundamental frequency in it that you cant see for the f because there is no fundamental frequency. SO these, the envelope between f, i, and u is rather different.  So lets see how closely they are realted to each other for the same vowels, The same sound.  SO there are three segemnts taken from the same u and you can see that these spectra look rather alike. So not the spectra itself, but the envelope. They are rather similar, so as I said before, the envelope is basicall what we do with our vocal tract and this carries the information so if the envelope looks more or less like this then it will sound like a you, while the envelope her in blue would sound like an f.  SO how do obtain the envelope?  Well it would be possible to just look at the wideband spectrogram and you basically have the envelope already.  But it changes rather quickly, so here  these horizontal lines, its more energy, less energy, more energy, so its not very robust, so there are more advanced techniques like LPC,Cepstrum)  The bottom line here is that most of the time, we will be using narrowband analysis, meaning that we use signal segments between 16 and 32ms and we will obtain the spectral envelop not from the wideband, spectrum, but via other methods that will be coming later.
