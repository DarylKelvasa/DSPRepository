% CHAPTER 3 LESSON 3
\clearpage
\section{Discrete Fourier Transform (DFT)}
\label{Discrete Fourier Transform (DFT)}

 The discrete FOurier transform is what we usually work with.  The discrete time FOurier transform that we defined before but the problem here is that if you in  practice you cannot compute it because it is defined as an infinite sum over the signal that you are looking at and in practice you cannot compute an infinite sum, it would take forever literally.  SO what you would have to do is in practice is take a finite time sequence and in a similar way to how we understood sampling again you can think about what does that mean if you chop a certain segment from a signal. And this you can model by multiplying the signal with a rectangular window.
 
 So this would be our time domain signal, with a certain spectrum which can be periodic becasue we have a discrete time fourier transform so this is a discrete signal in time domain giving a continuous and periodic spectrum.  We multiply this with a rectangular window function and this rectangular window function corresponds to a synch function in the frequency domain.  Meaning that if we multiply the two, what you get is a convolutuion of the spectrum with a synch function and this convolution means that you smear your spectrum in the specectral domain,  Meaning that the spectral content will be smeared which also means that your frequency resolution decreases for instance if you have two closely spaced sinuosids and you want to distinguish between them in the frequency domain you need this synch function to be narrow enough so that two signals do not interfere with each other. So lets say 
you have two sinusoidal signals that correspond to two deltas in the spectral domain.  SO we have x1 and x2 which are the sums of two sinusoids. So  now this figure is no longer infintelky log as we would beed it for the discret time FOurier transfom, but we would chop it somewhere.  That means that we will convolve the spectrum with a synch function and then we have a synch function here and a synch function there and you will add the sum of them, meaning that if the synch function gets too broad, then in the resulting spectrum you might get something that looks like this.  Where you are not able to see the two peaks.  This happens if you choose the rectangular window too small.  So in order to increase your spectral resolution you would need more data, in a sense, to find the two sinusoidal components. This is what we learn in this simple example, the trick is that we must know that a multiplecation in the time domain corresponds to convolution in the frequency domain and the fact that a rectangualr function in time domain corresponds to a synch function in the frequency domain .

So now we understand the influence of cutting the spectrum so what do we have here. If we sample the freequency spectrum with sampling period 1/nT then the final time sequence n samples will periodically repeated without and overlap, time domain aliasing. SO what you see here is that the what we are trying to derive here is that the discrete fourier tansform representaion. because in the representaion that we had before, the discret time fourioer  transform,  we had a discrete time representaiton but a continuous frequency representaion now we try to understand what happens if we sample my spectrum, it means that in time domain, my signal will be periodically repeated in the same way that we derived the frequency domain representation and out sampling theorem.

And what this basically means is thta if we discretize oiur spectrum, alos our time domain signal will be repeated and if then also take a rectangular windoww a certain length N, the same way that we did witjh the sampling theorem, we have to take care that when I periodically repeat my spectrum, that these two segments do not overlap. And that correpsonds to how I sample the spectrum in the frequency domain. SO what happens basically, is if you have a discret fourier tansform analysis, that first of allyou look at a windowed part of your time domain signal, then you sample your signal in the frequewncy domain and that corresponds to a periodica represntation of you signal.  If you do use the dft in matlab, then we only look at part of it, the first n sammples both in time domain and in frequency domain, but what you should keep in mind at least in the theory part, that these segemnts will be periodically repeated which is becaue a discrretized signal in time domain correpsonds to a periodic spectrum and a discret frequency domain signal corresponds to a periodic time domain signal. 

And then we have now the discrete fourier transform defintiom where we have bnoth a discrete time domain signal with sample index n and a discrete spectrum with frequency bin index k. And well for the discrete fourier transform, this is what we will mostlay use, for in speech analysis for in this lecture, it'll again have all of these propertioes, Linearity.

In practice, we have fast ways of computing the discrete fourier transform, for instance, the fast fourier transform.  And this is one of the reasons why the discrete fouriertransform is a very often used  spectral representation for speech signals or for audio processing in general  because we have this fast fourier trandform  which is computationally efficient and allows for a fast representaiton of the spectrum of a asignal. And this is one of the basic tools that we use in speech processing. 

Windowing corresponds to the fact that if you chop the signal, it corresponds to the convolution of the function wioth a synch function. For instance, if you have a sinusiod then this would ideally correspond to only one peak in the spectrum, so this  a DFT representaion.  And if the sinusoid fits exactly the window that  youre looking at then also, in the DFT domain, you would see one peak.  Howevre if this is not the case, so again we would have 16 samples inthe time domian, but now the frequency a bit diffeent,