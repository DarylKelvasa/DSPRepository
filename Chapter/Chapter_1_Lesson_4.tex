% CHAPTER 1 LESSON 4
\clearpage
\section{Hearing}
\label{Hearing}

The ear, ruglhly speaking consists of different parts, the outer ear, the middle ear and the inner ear. the outer ear consistrs of the pinna the ear canal and the ear drum and the middle ear consist of the jkahsfda stapus which work like a leather.  The inner ear consists of the cochlea which is where we perceive sound and is where the auditory nerve is attached. Sound aves travel through the ear canal and cause the ear drum to vibrate and we have this lever here so we transistion from the large area to the samall area and here we would have then the travelling wave travelling through the cochlea. Here we would have the basilarmembrane.

The intersting thing is that the basilar membrane does the frequency to place transformation meaning that we have a travelling wave that will have its peak at certain point in space, for instance if we have a tone a 3500hz then it would have its maximum at a certain point if we have a high frequency tone the maximum would closer to the base of the badilar membrane so the point where the stapus, while if we have lower frequenceies , the resonances will be toward the apex.  so the tip of the cochlea and thats very imporatn becasue it means that we also perceive sound in the frequency domain m,eaning that humans perform a frequency analysis of the signal . Thats also why this frequency analysis that we use is so natural for us because it is also how we perceive sound .

If we look a little closer, at a cut through the cochlea we will see these tubes that wind up.  The interesting part is the basilar mnembrance and the ortgan of corti  where we have the hair cells.  What happens is that there will be a travelöling wave that will cause this part to move against the tectortial membrnce so that we will have the movemnet of the hair cells and they will then fire through the auditopry nerve and then the brain will have a percepotion of sound .

SO this is the auditory sensation area for a normal hearing person, so we have acertain threashlod of hearing here we see the frequecny axis on a logarithmic scaling. So we can see that at lower frequencies, we need more energy to perceive sound than at higher frequencies up to a caertain order.  Frequencies below 20Hz we are not able to perceive, at least not throught the cohclea and also at higher frequencies, we have problems as we get older. At the top is the level of pain. If we hear a sound at a level higher than the line for an extenden period of time, we will start to have hearing damage. We can see that speech fills a certain range in the auditory senstaion area.   Interestingly, the area where all the formants are is also the area where our ear is most sensitive. Quite conveninet It means that our ear is optimized for speech perception. (or the other way around) We see taht music has a wider range in frequency.  If you listen to the radio we hear that it sounds difffernt but we can still understand everything that is being said, this is because we still transmit the important formants even though the music range is much wider.

So what happens if we are hearing impaired or if we experience a sernsory neural hearing loss? So what happens approximately, or simply speaking is that the level of pain approx dont change, while the threshlod of hearing is increased.  We then need to amplify soft sounds but if we would just linearly amplify the level of sounds, we would o that beyond the level of pain and thats why we need compression algorithm in hearing aids, so tha twe amplify soft sounds more than loud sounds.  This decreases the SNR and therefore requires some noise reducxtion to enhance the signal to our hearing aid. 

Its interesting to note that there are different types of hearing losses.  The conductivbe hearing loss is the one can be traeated more easily that a senorsneural hearing loss  because it means that the sound is not properly conducted by the outer ear or the middle ear. The bomes could stiffen The sound is still perceiveable, but attentuatoed which also means that you can also treat it rather well with a hearing aid by amplyfying the sound. Sensorineural hearing loss is for instance what we have with th age related hearing loss so we get older then our hair cells die. They can also die because of trauma for instance if you are in very noisy or loud environment or if there is a gunshot. This can dmage the hair cells and then there is nothing that we could do at this time.  This sensory neural hearing loss is often accompanied by tinnitus where when one doesnt hear well, instead they hear a ringing. And then what else happens is that soft sounds are too soft and loud sounds are too loud which basically measn that we have this reduced area in the auditory sensation map. Whats also problematic is that the sensory neural hearing loss goes along with a decrease in frequency resolution meaning thta even if you do a audtitory test, you can still preform well to some extent, but still you have trouble when you want to perceive speech in noise cause the frequency resolution of your auditory perception is decreased. And this means that we cannot so well distinguish between speech commands. And that means that we have a decreased speech understanding in noise and for this reason again noise reduction is a more important aspect.

Current hearing aids commonly implement multi microphones. In this model there are three different microphones that are channeled to an analysis filter bank where a time frequency analysis is performed similar to what we do in the cochlea. Then we we do a directional processing with the microphone. Because there are multiple microphones, we are able to cancel out the sounds that come from a specific direction while keeping the sounds that come from the fron for instance unattentuated. This is also done in hearing aids. Basically you can choose whom you want to listen to by lookingat that person. However this is not a very narrow beam, but a very broad beam.We can make this beam more narrow, we could have, for instance a binaural hearing aid where we have two hearing aids that communicates meaning that we have microphones placed further apart because these microphones are only on one hearing aid so if we pull them apart, we can produce a more narrow beam and this is  well technology that is evolving, but we must transmit this information to the other hearing aid.  This can be done with a wire, or wirelessly, but then we have a large energy consumption.  But in the simplest case, we transmit control information, for instance volume control makes more sense if both hearing aids are in the same state. The same holds for calssificaton of the background noise.  Many hearing aids have auditory scene analysers where they for instance let say if you are in a noisy environment and we want to do speech communication then you want to turn on the noise reduction in order to perceive speech better.  But if you are at a concert, you listen to some music then you dont want noise reduction depending on your tastes i music. :) SO that would mean that the hearing aids would have certain algorithms that control certain parameters of the auditory processing. There is also then a feedback cancellation stage where we cancel out feedback loops. 