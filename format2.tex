%----------------------------------------------------------------------------------------
%	VARIOUS REQUIRED PACKAGES
%----------------------------------------------------------------------------------------
%\newcommand{\figref}[1]{\figurename~\ref{#1}}
%\newcommand{\tabref}[1]{Table~\ref{#1}}
%\newcommand{\secref}[1]{Sec.~\ref{#1}}
%\renewcommand{\eqref}[1]{Eq.~(\ref{#1})}
%
%\newcommand{\kommentar}[1]{\-\marginpar[\raggedleft\footnotesize \textsf {#1}]{\raggedright\footnotesize \textsf {#1}}}

%expected value
\newcommand{\E}[1]{\mathrm{E}{  \left(  #1 \right)  }}
\newcommand{\EY}[2]{\mathrm{E}_{#1}{  \left(  #2 \right)  }}

\newcommand{\mean}[1]{m_{#1}}
\newcommand{\autocorr}[1]{\varphi_\mathrm{#1#1}}
\newcommand{\crosscorr}[1]{\varphi_\mathrm{#1}}
\newcommand{\crosscorre}[1]{\widehat{\varphi}_\mathrm{#1}}
\newcommand{\Autocorr}[1]{\Phi_\mathrm{#1#1}}
\newcommand{\Crosscorr}[1]{\Phi_\mathrm{#1}}
\newcommand{\Autocorrbf}[1]{\boldsymbol{\Phi}_\mathrm{#1#1}}
\newcommand{\Crosscorrbf}[1]{\boldsymbol{\Phi}_\mathrm{#1}}
\newcommand{\autocorre}[1]{\widehat{\varphi}_\mathrm{#1#1}}
\newcommand{\corrvec}[1]{\mathbf{\varphi}_\mathrm{#1}}
\newcommand{\corrmat}[1]{\mathbf{R}_\mathrm{#1}}
\newcommand{\psd}[1]{\Phi_{#1}}

% function name
\newcommand{\funcname}[1]
{\mathrm{#1}}

% estimate
\newcommand{\estimation}[1]
{\ensuremath{\widehat{#1}}} % symbol for estimation

%smoothed quantity
\newcommand{\smoothB}[1]{\widetilde{#1}}
\newcommand{\smooth}[1]{\bar{#1}}

\renewcommand{\vec}[1]{\boldsymbol{#1}} % vector

% variance of ... \var{x} \sigma
\newcommand{\var}[1]{\funcname{var}\!\lefta{#1}} 
\newcommand{\cov}[1]{\mbox{cov}\!\lefta{#1}}

% e to the power of ... ( e^{} )
\newcommand{\epow}[1]{\funcname{e}^{#1}}

% argmax operator
\newcommand{\argmax}[1]{
\underset{#1}{\funcname{argmax}}}



% frequency domain
\newcommand{\numfreq}%
	{\ensuremath{N} }   		%number of samples per segment
\newcommand{\frameshift}%
	{\ensuremath{R} }	% frame shift
\newcommand{\freqi}%
	{\ensuremath{k} }   		% frequency index
\newcommand{\framei}%
	{\ensuremath{\ell} }    		% frame index
\newcommand{\ff}{_\freqi\!\lefta{\framei}} % function of (frequ,frame)
\newcommand{\ffprev}{_\freqi\!\lefta{\framei-1}} % function of (frequ,frame-1)

%real and imaginary part
\newcommand{\real}[1]{\mathrm{Re}\leftc{#1}} % Re{.}
\newcommand{\imag}[1]{\mathrm{Im}\leftc{#1}} % Im{.}



%===========================================================
%=== SIGNALS/SPECTRA
%===========================================================
\newcommand{\Xc}%  complex random variable
        {\ensuremath{X} }  % complex spectrum of speech S
\newcommand{\Xcf}%
       {\ensuremath{\Xc_\freqi} } 
\newcommand{\xrf}%
       {\ensuremath{\xr_\freqi} }
\newcommand{\xr}%
       {\ensuremath{x} }  
\newcommand{\Sc}%  complex random variable
        {\ensuremath{S} }  % complex spectrum of speech S
\newcommand{\Scf}%
       {\ensuremath{\Sc_\freqi} } 
\newcommand{\Scff}%
       {\ensuremath{\Sc\ff} } 
\newcommand{\sr} %realization
        {\ensuremath{s} }
\newcommand{\srf}%
       {\ensuremath{\sr_\freqi} } 
\newcommand{\srff}%
       {\ensuremath{\sr\ff} } 
\newcommand{\ssig}
       {\ensuremath{s[n]} }
\newcommand{\Sa}%  magnitude of complex random variable
        {\ensuremath{A} }  % 
        \newcommand{\sar}%  magnitude of complex realization
        {\ensuremath{a} }  % 
\newcommand{\Saf}%
       {\ensuremath{\Sa_\freqi} } 
\newcommand{\Saff}%
       {\ensuremath{\Sa\ff} } 
\newcommand{\Sphase}% speech phase random variable
        {\ensuremath{\Phi} }  % speech phase random variable
\newcommand{\Sphasef}% speech phase random variable
        {\ensuremath{\Sphase_\freqi} }  % speech phase random variable
 \newcommand{\sphase}% speech phase realization
        {\ensuremath{\phi} }  % speech phase realization\newcommand{\Nc}%
\newcommand{\Nc}%  complex random variable
	{\ensuremath{N} }
\newcommand{\Ncf}%
	{\ensuremath{\Nc_\freqi} }	
\newcommand{\Ncff}%
	{\ensuremath{\Nc\ff} }
\newcommand{\nr}%
	{\ensuremath{n} }
\newcommand{\nrf}%
	{\ensuremath{\nr_\freqi} }	
\newcommand{\nrff}%
	{\ensuremath{\nr\ff} } 		
\newcommand{\nsig}
       {\ensuremath{n[n]} }

\newcommand{\Yc}%
	{\ensuremath{Y} }
\newcommand{\Ycf}%
       {\ensuremath{\Yc_\freqi} } 	
\newcommand{\Ycff}%
	{\Yc\ff}
\newcommand{\yr}%
	{\ensuremath{y} }
\newcommand{\yrf}%
       {\ensuremath{\yr_\freqi} } 	
\newcommand{\yrff}%
	{\yr\ff}
\newcommand{\ysig}
       {\ensuremath{y[n]} }
\newcommand{\yframe}
       {\ensuremath{\underline{y}_\framei} }
\newcommand{\sframe}
       {\ensuremath{\underline{s}_\framei} }
\newcommand{\Ya}%  magnitude of complex random variable
        {\ensuremath{R} }  %
\newcommand{\Yaf}%  magnitude of complex random variable
        {\ensuremath{R_\freqi} }  %  
        \newcommand{\yar}%  magnitude of realization
        {\ensuremath{r} }  % 
       \newcommand{\Yphase}% noisy phase random variable
        {\ensuremath{\Theta} }  % noisy phase random variable
\newcommand{\Yphasef}% noisy phase random variable
        {\ensuremath{\Yphase_\freqi} }  % noisy phase random variable
 \newcommand{\yphase}% noisy phase realization
        {\ensuremath{\theta} }  % noisy phase realization
% Cepstrum
\newcommand{\sq}%
	{\ensuremath{s} }  	
\newcommand{\sqcf}%
	{\sq\cf}
\newcommand{\sqc}%
	{{\sq_\cepsi}}
  \newcommand{\sqtcf}%
	{\widetilde{\sq}\cf}
    \newcommand{\sqtc}%
	{\widetilde{\sq}_\cepsi}			

\newcommand{\nq}%
	{\ensuremath{n} }  	
\newcommand{\nqcf}%
	{\nq\cf}	
  \newcommand{\nqtcf}%
	{\widetilde{\nq}\cf}	
\newcommand{\nqtc}%
	{\widetilde{\yq}_\cepsi}

\newcommand{\yq}%
	{\ensuremath{y} }  	
\newcommand{\yqcf}%
	{\yq\cf}		
\newcommand{\yqtcf}%
	{\widetilde{\yq}\cf}	
\newcommand{\yqtc}%
	{\widetilde{\yq}_\cepsi}

% estimation 

\newcommand{\stime}{s}
\newcommand{\ytime}{y}
\newcommand{\ntime}{n}
\newcommand{\setime}{\widehat{s}}

\newcommand{\Sec}%
        {\ensuremath{\estimation{\Sc}} }    % estimation of S
\newcommand{\Seff}{\Sec\ff}            % estimation of S (frequ,frame)
\newcommand{\Secf}{\Sec_\freqi}            % estimation of S (frequ,frame)
\newcommand{\Seffprev}{\Sec\ffprev}            % estimation of S (frequ,frame)

\newcommand{\seff}%
        {\ensuremath{\estimation{\sr}\ff } }    % realization of estimate
\newcommand{\ssigest} %signal estimate
       {\ensuremath{\estimation{s}[n]} }
\newcommand{\sframeest}
       {\ensuremath{\estimation{\underline{s}}_\framei} }%frame estimate
\newcommand{\Saestff}%
       {\ensuremath{\estimation{\Sa}\ff} } 

\newcommand{\Saest}%
       {\ensuremath{\estimation{\Sa}} } 

%===========================================================
%=== DCT / CEPSTRUM
%===========================================================
\newcommand{\dct}[1]{\funcname{DCT}\leftc{#1}}
\newcommand{\idct}[1]{\funcname{IDCT}\leftc{#1}}
\newcommand{\cepsi}{q}
\newcommand{\cf}{_\cepsi\!\lefta{\framei}}
\newcommand{\cfprev}{_\cepsi(\framei-1)}
\newcommand{\pitch}{f_0}

%===========================================================



%===========================================================
%=== STATISTICS
%===========================================================

\newcommand{\pdfg}[2]%
	{\ensuremath{\,p_{#1}\!\!\lefta{#2}} }	% general pdf p.(.)
\newcommand{\pdfs}[1]%
    {\ensuremath{{\pdfg{s}{#1}} }}    	        % ps(.)
\newcommand{\pdfn}[1]%
    {\ensuremath{{\pdfg{n}{#1}} }}     	        % pn(.)
\newcommand{\pdf}[1]{\pdfg{\,}{#1}}      	% p(.)
% conditional pdf
\newcommand{\pdfcond}[2]{\pdf{#1\cnd#2}}
\newcommand{\pdfh}[1]{\pdfg{h}{#1}}      	% histogram

\newcommand{\sig}[1]{\sigma_{\!\scriptscriptstyle \mathrm{#1}}}%
\newcommand{\sigS}{\sigma_{\!\scriptscriptstyle \mathrm{S}}}%  deviation of speech ss
\newcommand{\sigSf}{\sigma_{\!{\scriptscriptstyle \mathrm{S}},\freqi}}% 
\newcommand{\sigSfp}{\sigma_{\!{\scriptscriptstyle \mathrm{S}},\freqi-1}}% 
\newcommand{\sigSfpp}{\sigma_{\!{\scriptscriptstyle \mathrm{S}},\freqi+1}}% 
\newcommand{\sigSsq}{\sigma^2_{\!\scriptscriptstyle \mathrm{S}}}%
\newcommand{\sigSsqf}{\sigma^2_{\!{\scriptscriptstyle \mathrm{S}},\freqi}}%  
\newcommand{\sigSsqff}{\sigma^2_{\!{\scriptscriptstyle \mathrm{S}},\freqi}(\framei)}%  
\newcommand{\sigSe}{\estimation{\sigma}_{\!{\scriptscriptstyle \mathrm{S}}}}
\newcommand{\sigSef}{\estimation{\sigma}_{\!{\scriptscriptstyle \mathrm{S}},\freqi}}% 
\newcommand{\sigSeSq}{\estimation{\sigma^2_{\!{\scriptscriptstyle \mathrm{S}}}}} % sigmaS squared estimate

\newcommand{\sigN}{\sigma_{\!\scriptscriptstyle \mathrm{N}}}% deviation of noise sn
\newcommand{\sigNf}{\sigma_{\!{\scriptscriptstyle \mathrm{N}},\freqi}}%
\newcommand{\sigNsqf}{\sigma^2_{\!{\scriptscriptstyle \mathrm{N}},\freqi}}%  
\newcommand{\sigNsq}{\sigma^2_{\!\scriptscriptstyle \mathrm{N}}}% 
\newcommand{\sigNe}{\estimation{\sigma}_{\!{\scriptscriptstyle \mathrm{N}}}}
\newcommand{\sigNef}{\estimation{\sigma}_{\!{\scriptscriptstyle \mathrm{N}},\freqi}}% 
\newcommand{\sigNeSq}{\estimation{\sigma^2_{\!{\scriptscriptstyle \mathrm{N}}}}}
\newcommand{\sigNefSq}{\estimation{\sigma^2_{\!{\scriptscriptstyle \mathrm{N}},\freqi}}}% 
\newcommand{\sigNsqff}{\sigma^2_{\!{\scriptscriptstyle \mathrm{N}},\freqi}(\framei)}%  

\newcommand{\sigXfSq}{{\sigma^2_{\!{\scriptscriptstyle \mathrm{X}},\freqi}}}% 


\newcommand{\sigY}{\sigma_{\!\scriptscriptstyle \mathrm{Y}}}% 
\newcommand{\sigYf}{\sigma_{\!{\scriptscriptstyle \mathrm{Y}},\freqi}}% 

\newcommand{\sigSsqfML}{\sigma^{2,\mathrm{ML}}_{\!\scriptscriptstyle \mathrm{S},\freqi}}%
\newcommand{\sigSsqML}{\sigma^{2,\mathrm{ML}}_{\!\scriptscriptstyle \mathrm{S}}}%        

\newcommand{\sigSefSq}{\estimation{\sigma^2_{\!{\scriptscriptstyle \mathrm{S}},\freqi}}}% 

                                        
\newcommand{\prob}[1]{P\!\lefta{#1}}  % probability P(.)
\newcommand{\probc}[2]{P\!\lefta{#1 \cnd #2}}
\newcommand{\probcRV}[4]{P_{#1 \mid #2}\!\lefta{#3 \mid #4}}
                                        % conditional probability P(.)
\newcommand{\cnd}{\bigm|}               % condition |

\newcommand{\hyp}[1]{{\cal H}_{#1}}     % hypothesis
\newcommand{\hypf}[1]{{\cal H}_{#1,\freqi}}     % hypothesis
\newcommand{\hypff}[1]{{\cal H}_{#1,\freqi}(\framei)}     % hypothesis

\newcommand{\est}[1]{E\leftc{#1}}       % estimate E{.}
\newcommand{\estc}[2]{E\leftc{\left.#1\right|#2}}
                                % conditional estimate E{.|.}
\newcommand{\varc}[2]{\,\mbox{\it Var}\leftc{\left.#1\right|#2}}
                                % conditional variance Var{.|.}

%===========================================================
%=== SNR
%===========================================================
\newcommand{\snr}%
    {\ensuremath{\xi} }% a priori SNR
\newcommand{\snrf}%
    {\ensuremath{\snr_\freqi} }%     
\newcommand{\snrff}%
    {\ensuremath{\snr\ff} }% 

\newcommand{\snrg}[1]{\snr_{#1}}% general 
\newcommand{\snrmin}{\snrg{\mathrm{min}}}% 

\newcommand{\snre}{\estimation{\snr}}% estimated a priori SNR
\newcommand{\snreff}{\snre\ff}% 
\newcommand{\snrefprev}{\snre\ffprev}% 
                                    % (frequ,frame-1)

\newcommand{\snrpost}{\zeta} % aposteriori SNR
\newcommand{\snrpostRV}{Z} % random variable: HERE WE NEED ANOTHER NAME
\newcommand{\snrpostRVf}{Z_\mathrm{\freqi}} % random variable: HERE WE NEED ANOTHER NAME
\newcommand{\snrpostf}{\snrpost_\freqi}
\newcommand{\snrpostff}{\snrpost\ff}
\newcommand{\snrposte}{\estimation{\snrpost}}
\newcommand{\snrpostef}{\estimation{\snrpost}_\freqi}
\newcommand{\snrposteff}{\estimation{\snrpost}\ff}

\newcommand{\snrpostth}{{\snrpost_{th}}}

\newcommand{\segimp}{\Delta\segsnr} % improvement of seg. SNR

%===========================================================
%=== FUNCTIONS
%===========================================================
\newcommand{\gain}{g}
\newcommand{\Gain}{G}
\newcommand{\Gainf}{\Gain_\freqi}
\newcommand{\Gainff}{\Gain\ff}
\newcommand{\Gainhypf}[1]{\Gain_{\hypnof{#1},\freqi}}
\newcommand{\Gainmin}{\Gain_{\mb{min}}}
\newcommand{\Gainlim}{\widetilde{\Gain}}
\newcommand{\Gainlimf}{{\widetilde{\Gain}_\freqi}}


\newcommand{\segsnr}%
	{\ensuremath{\funcname{SNR}_{seg}} }% segmental SNR
\newcommand{\mos}{MOS}% Mean-Opinion Score value

%===========================================================
%=== BRACKETS
%===========================================================
\newcommand{\lefta}[1]{\left(#1\right)}
\newcommand{\leftb}[1]{\left[#1\right]}
\newcommand{\leftc}[1]{\left\{#1\right\}}


%===========================================================
%=== OPERATORS
%===========================================================

\newcommand{\erf}{\funcname{erf}}
\newcommand{\erfa}[1]{\,\mbox{erf}\!\lefta{#1}}
\newcommand{\erfi}[1]{\,\mbox{erfi}\!\lefta{#1}}
\newcommand{\erfca}[1]{\,\mbox{erfc}\!\lefta{#1}}
\renewcommand{\exp}{\funcname{exp}} 
\newcommand{\expa}[1]{\exp\!\lefta{#1}} 
\newcommand{\expb}[1]{\exp\!\leftb{#1}}    
\newcommand{\expc}[1]{\exp\!\leftc{#1}}
\newcommand{\expp}[1]{\mathrm{e}^{#1}}
\newcommand{\loga}[1]{\,\funcname{log}\!\lefta{#1}}
\newcommand{\loge}[1]{\,\funcname{log}\!\lefta{#1}}
\newcommand{\logd}[1]{\,\funcname{log}_2\!\lefta{#1}}
\newcommand{\logt}[1]{\log_{10}\!\lefta{#1}}
\newcommand{\logta}[1]{\logt\lefta{#1}}
\newcommand{\gammaf}[1]{\,\Gamma\!\lefta{#1}}% gamma function
\newcommand{\abs}[1]{\left|#1\right|}
\newcommand{\sgn}[1]{\,\mbox{sgn}\!\left(#1\right)}
\renewcommand{\jmath}{\mathrm{j}}
\newcommand{\maxa}[1]{\max\!\lefta{#1}}
\newcommand{\mina}[1]{\min\!\lefta{#1}}


\newcommand{\idftexp}[3]{1/\numfreq \sum_{#2 =0 }^{\numfreq-1} #3 \, \expp{\jmath 2\pi #2 #1 /\numfreq}}
\newcommand{\dftexp}[3]{            \sum_{#2 =0 }^{\numfreq-1} #3 \, \expp{- \jmath 2\pi #1 #2 /\numfreq}}
\newcommand{\dft}[1]%
	{\ensuremath{\funcname{DFT}\!\leftc{#1} }} %_{\dftlength}} }
	% DFT transform symbol DFT{.}_M
\newcommand{\idft}[1]%
	{\ensuremath{\funcname{IDFT}\!\leftc{#1}} } %_{\dftlength}} }

\newcommand{\intinf}{\int\limits_{-\infty}^{+\infty}}
\newcommand{\Euler}{\mbox{  \boldmath$C$}}


%===========================================================
%=== possible italics
%===========================================================
\newcommand{\apriori}{\emph{a priori }}
\newcommand{\Apriori}{\emph{A priori }}
\newcommand{\aposteriori}{\emph{a posteriori }}
\newcommand{\Aposteriori}{\emph{A posteriori }}
\newcommand{\eg}{{e.g.\ }}
\newcommand{\ie}{{i.e. }}
\newcommand{\etal}{{et~al. }}



%===========================================================
%=== units
%===========================================================
\newcommand{\ms}{\,\mathrm{ms}}
\newcommand{\Hz}{\,\mathrm{Hz}}
\newcommand{\kHz}{\,\mathrm{kHz}}

% decibel
\newcommand{\dB}{\,\mathrm{dB}}


%speech Enhancement
\newcommand{\absfshift}{L}

\newcommand{\biascorr}{\mathcal{B}}
\newcommand{\indicator}{b}

\newcommand{\alphadd}{\alpha_{\mathrm{dd}}}

\newcommand{\srate}{f_\mathrm{s}}
\newcommand{\lag}{\lambda}
\newcommand{\timei}{n}
\newcommand{\timeiseg}{\timei}
\newcommand{\alttime}{m}
\newcommand{\orderq}{q}
\newcommand{\orderp}{p}
\newcommand{\expjO}{\expp{\jmath \Omega} }



% SPP
\newcommand{\aprioriSPP}{\prob{\hyp{1}}}%{\rho}
\newcommand{\aprioriSPPhat}{\widehat{P}(\hyp{1})}
\newcommand{\aprioriSPPhatf}{\widehat{P}(\hypf{1})}
\newcommand{\aprioriSPPhatff}{\widehat{P}(\hypf{1}(\framei))}
\newcommand{\aprioriSAP}{\prob{\hyp{0}}}
% \newcommand{\apostSPP}{{\mathcal{P}_\freqi}}
\newcommand{\apostSPP}{\probc{\hyp{1}}{\Yc}}
\newcommand{\apostSPPRV}{\probcRV{\hyp{1}}{\Yc}{\hyp{1}}{\yc}}
\newcommand{\apostSPPff}{\probc{\hyp{1}}{\Ycff}}
\newcommand{\apostSPPfr}{\probc{\hyp{1}}{\Yc(\framei)}}
\newcommand{\apostP}{\mathcal{P}}
\newcommand{\apostSAP}{{\probc{\hyp{0}}{y}}}
\newcommand{\apostSAPf}{\apostSAP}



\newcommand{\Per}{\mathrm{Per}}
\newcommand{\Perf}{\mathrm{Per}_\freqi}
\newcommand{\per}{\mathrm{per}}
\newcommand{\perf}{\mathrm{per}_\freqi}


\newcommand{\samplingPeriod}{T_{s}}
\newcommand{\samplingFreq}{f_{s}}
\newcommand{\sampTime}{t}
\newcommand{\sampIndex}{n}
\newcommand{\fourierSym}{\mathfrak{F}}
\newcommand{\deltaPulse}{\delta}
\newcommand{\contFunc}{x}
\newcommand{\discFunc}{d}
\newcommand{\sinCoef}{a}
\newcommand{\cosCoef}{b}




%--- from PPre format
\usepackage[dvips]{graphicx}            % für zwei Bilder in Pelton Versuch
%\usepackage{graphicx}                   % neue Graphik Umgebung
\usepackage{color}
% \usepackage{epsfig}                   % war mal ... 
\usepackage{wrapfig}                    % JJT: for wrapping text around figs.
\usepackage{floatflt}                      % wof"ur? - irgendwas f"ur ToC
\usepackage{fancyhdr}                   % f�r all die Kopf- & Fu�zeilen
\usepackage{german}                     % Grundeinstellungen
\usepackage{times,a4wide}
% \usepackage[applemac]{inputenc}        % Umlaute von Tastatur
% \usepackage[microsoft]{inputenc}       % Umlaute von Tastatur
\usepackage{setspace}                   % f"ur `doublespacing' z.B.
\usepackage{versions}                    % `comments' u.�.
%\usepackage{indentfirst}               % auch erster Absatz erh"alt Einzug
%\usepackage{latexsym,ae}
\usepackage{verbatim}
%\usepackage{babel}
\usepackage[section] {placeins}					%to (hopefully) avoid too many unprocessed floats
\usepackage{epic}                       % f�r Spektrum Skizze
                                        %       -- mu� vor pictex stehen
%%%%%%%\usepackage{pictex}                     % l"adt `pictex' f"ur einige Pumpen Bilder
                                        % `pictexwd' ist verbessertes Modul,
                                        % l"auft aber nicht auf Mac, uk
\usepackage[squaren]{SIunits}           % /metre u.a. im mathematikmodus

\usepackage[latin9]{inputenc}		% für die Umlaute von Tastatur

%\renewcommand{\familydefault}{\sfdefault} %Set default font to sans serif 
		
\selectlanguage{english}            % Hauptsprache: britisches Englisch

\newcommand{\version}{30.09.07}    % damit etwas in der Variablen steht ...
\newcommand{\name}[1]{\textsc{#1}}
%--- /from PPre format

\usepackage{titlesec} % Allows customization of titles


%\titleformat{\part}[block]{\normalfont\sffamily}{}{}{}{} %/rk:doesn't work just like this

\usepackage[T1]{fontenc}
%\usepackage[latin1]{inputenc} %wird in format_PPRE_reader definiert
%\usepackage{graphicx} % wird in format_PPRE_reader definiert

\usepackage{xcolor} % Required for specifying colors by name
\definecolor{chapcolor}{RGB}{243,102,25} % Define the orange color used for highlighting throughout the book

% Font Settings
\usepackage{avant} % Use the Avantgarde font for headings


\usepackage{graphicx} % Required for including pictures

\usepackage{lipsum} % Inserts dummy text

\usepackage{tikz} % Required for drawing custom shapes
\usetikzlibrary{arrows,calc,positioning}
%\usepackage{subfig}

\usepackage{pgfplots} % Required for drawing custom shapes
\pgfplotsset{width=7cm,compat=1.8}
%a) add the path to the /bin directory where u installed gnuplot to the windows "path" variable (und%er control panel -> system -> advanced -> environment variables)
%b) make sure there is a COPY of the executable in the /bin folder called "gnuplot.exe" (what i had to do is copy "pgnuplot.exe" to "gnuplot.exe"... wtf!?)
%c) add "-enable-write18" to the command line options of the output profile ur using in texniccenter

\usepackage[labelfont=bf,textfont=small]{caption}
% will apply to all subcaptions
\usepackage[labelfont=bf,textfont=small,singlelinecheck=on,justification=raggedright]{subcaption}

\usepackage[english]{babel} % English language/hyphenation

\usepackage{enumitem} % Customize lists
\setlist{nolistsep} % Reduce spacing between bullet points and numbered lists

\usepackage{booktabs} % Required for nicer horizontal rules in tables

\usepackage{eso-pic} % Required for specifying an image background in the title page

\usepackage{floatrow} % Required for setting the captions beside the image in a figure environment, generates error with double captions

%Bibliograhy packages
\usepackage{csquotes} %Required by bibtex
% \usepackage[style=alphabetic, bibstyle=alphabetic, backend=biber]{biblatex}
% \usepackage[nottoc, section, numbib, numindex]{tocbibind}  % fuegt
\usepackage[nottoc, numbib, numindex]{tocbibind}  % fuegt Literaturverzeichnis ins Inhaltsverzeichnis ein

\usepackage[colorlinks=true,
linkcolor=black,
citecolor=black,
filecolor=black,
urlcolor=black,
bookmarks=true,
bookmarksopen=true,
bookmarksopenlevel=1,
plainpages=false,
pdfpagelabels=true]{hyperref} %rereferencing


 \DeclareFontFamily{U}{wncy}{}
    \DeclareFontShape{U}{wncy}{m}{n}{<->wncyr10}{}
    \DeclareSymbolFont{mcy}{U}{wncy}{m}{n}
    \DeclareMathSymbol{\Sh}{\mathord}{mcy}{"58} 



% Index
\usepackage{calc} % For simpler calculation - used for spacing the index letter headings correctly
\usepackage{makeidx} % Required to make an index
\makeindex % Tells LaTeX to create the files required for indexing

% Define box and box title style
\tikzstyle{mybox} = [draw=red, fill=blue!20, very thick,
    rectangle, rounded corners, inner sep=0.25\linewidth, inner ysep=10pt]
\tikzstyle{fancytitle} =[fill=red, text=white]

%Remove Part before Apendix
%\renewcommand{\partname}{}

%\usepackage{morefloats}  % damit mehr floating  graphics verarbeitet werden k�nnen
%----------------------------------------------------------------------------------------
%	MAIN TABLE OF CONTENTS
%----------------------------------------------------------------------------------------

\usepackage{titletoc} % Required for manipulating the table of contents

\contentsmargin{0cm} % Removes the default margin

% Part text styling
\titlecontents{part}[1.25cm] % Indentation
{\addvspace{15pt}\large\sffamily\bfseries} % Spacing and font options for chapters
{\color{black!60}\contentslabel[\Large\thecontentslabel]{1.25cm}\color{black}} % Chapter number
{}  
{\color{black!60}\normalsize\sffamily\bfseries\;\titlerule*[.5pc]{.}\;\thecontentspage} % Page number

% Chapter text styling
%\titlecontents{chapter}[1.25cm] % Indentation
\titlecontents{chapter}[0.35cm] % Indentation
{\addvspace{15pt}\large\sffamily\bfseries} % Spacing and font options for chapters
{\color{chapcolor!60}\contentslabel[\Large\thecontentslabel]{1.25cm}\color{chapcolor}} % Chapter number
{}  
{\color{chapcolor!60}\normalsize\sffamily\bfseries\;\titlerule*[.5pc]{.}\;\thecontentspage} % Page number
% Section text styling
\titlecontents{section}[1.25cm] % Indentation
{\addvspace{5pt}\sffamily\bfseries} % Spacing and font options for sections
{\contentslabel[\thecontentslabel]{1.25cm}} % Section number
{}
{\sffamily\hfill\color{black}\thecontentspage} % Page number
[]
% Subsection text styling
\titlecontents{subsection}[1.25cm] % Indentation
{\addvspace{1pt}\sffamily\small} % Spacing and font options for subsections
{\contentslabel[\thecontentslabel]{1.25cm}} % Subsection number
{}
{\sffamily\;\titlerule*[.5pc]{.}\;\thecontentspage} % Page number
[] 

%----------------------------------------------------------------------------------------
%	MINI TABLE OF CONTENTS IN CHAPTER HEADS
%----------------------------------------------------------------------------------------

% Section text styling
\titlecontents{lsection}[0em] % Indendating
{\footnotesize\sffamily} % Font settings
{}
{}
{}

% Subsection text styling
\titlecontents{lsubsection}[.5em] % Indentation
{\normalfont\footnotesize\sffamily} % Font settings
{}
{}
{}
 
%----------------------------------------------------------------------------------------
%	PAGE HEADERS
%----------------------------------------------------------------------------------------

\usepackage{fancyhdr} % Required for header and footer configuration

\pagestyle{fancy}
\renewcommand{\chaptermark}[1]{\markboth{\sffamily\normalsize\bfseries #1}{}} % Chapter text font settings
\renewcommand{\sectionmark}[1]{\markright{\sffamily\normalsize\thesection\hspace{5pt}#1}{}} % Section text font settings
\fancyhf{}
\fancyhead[LO]{\rightmark} % Print the nearest section name on the left side of odd pages
\fancyhead[RE]{\leftmark} % Print the current chapter name on the right side of even pages
\renewcommand{\headrulewidth}{0.5pt} % Width of the rule under the header %
\addtolength{\headheight}{2.5pt} % Increase the spacing around the header slightly
\fancypagestyle{plain}{\fancyhead{}\renewcommand{\headrulewidth}{0pt}} % Style for when a plain pagestyle is specified
\fancyheadoffset{1cm}

% Removes the header from odd empty pages at the end of chapters
\makeatletter
\renewcommand{\cleardoublepage}{
\clearpage\ifodd\c@page\else
\hbox{}
\vspace*{\fill}
\thispagestyle{empty}
\newpage
\fi}

 \fancyfoot[LE,RO]{\sffamily\normalsize\thepage} % Font setting for the page number in the header
\renewcommand{\footrulewidth}{0.0pt} % Width of the rule above the footer %
\setlength{\footskip}{20mm}% Increase the spacing around the header slightly 



\fancyfoot[LO]{\sffamily\normalsize\bfseries Digital Speech Processing} % Print the nearest section name on the left side of odd pages
\fancyfoot[RE]{\sffamily\normalsize\bfseries Digital Speech Processing} % Print the current chapter name on the right side of even pages



\fancyfootoffset{1cm}%Lengthens the rule%

\setlength{\textheight}{235mm}% Increase the spacing around the header slightly 

%----------------------------------------------------------------------------------------
%	THEOREM STYLES
%----------------------------------------------------------------------------------------

\usepackage{mathtools,amsfonts,amssymb,amsthm} % For including math equations, theorems, symbols, etc

\newcommand{\intoo}[2]{\mathopen{]}#1\,;#2\mathclose{[}}
\newcommand{\ud}{\mathop{\mathrm{{}d}}\mathopen{}}
\newcommand{\intff}[2]{\mathopen{[}#1\,;#2\mathclose{]}}
\newtheorem{notation}{Notation}[chapter]

\newtheoremstyle{chapcolornum} % Theorem style name
{7pt} % Space above
{7pt} % Space below
{\normalfont} % Body font
{} % Indent amount
{\small\bf\sffamily\color{chapcolor}} % Theorem head font
{\;\;} % Punctuation after theorem head
{0.25em} % Space after theorem head
{\small\sffamily\color{chapcolor}\thmname{#1}
%{\small\sffamily\color{chapcolor}\thmname{#1}\thmnumber{\@ifnotempty{#1}{ }\@upn{#2}} % Theorem text (e.g. Theorem 2.1)
\thmnote{\ {\the\thm@notefont\sffamily\bfseries\color{black}--- #3.}}} % Optional theorem note
\renewcommand{\qedsymbol}{$\blacksquare$} % Optional qed square

\newtheoremstyle{blacknumex} % Theorem style name
{7pt} % Space above
{7pt} % Space below
{\normalfont} % Body font
{} % Indent amount
{\small\bf\sffamily} % Theorem head font
{\;\;} % Punctuation after theorem head
{0.25em} % Space after theorem head
%{\small\sffamily{\tiny\ensuremath{\blacksquare}}\ \thmname{#1}\thmnumber{\@ifnotempty{#1}{ }\@upn{#2}} % Theorem text (e.g. Theorem 2.1) %Removed black qed box in next line
{\ \thmname{#1}\thmnumber{\@ifnotempty{#1}{ }\@upn{#2}} % Theorem text (e.g. Theorem 2.1)

\thmnote{\ {\the\thm@notefont\sffamily\bfseries--- #3.}}} % Optional theorem note

\newtheoremstyle{blacknum} % Theorem style name
{7pt} % Space above
{7pt} % Space below
{\normalfont} % Body font
{} % Indent amount
{\small\bf\sffamily} % Theorem head font
{\;\;} % Punctuation after theorem head
{0.25em} % Space after theorem head
{\small\sffamily\thmname{#1}\thmnumber{\@ifnotempty{#1}{ }\@upn{#2}} % Theorem text (e.g. Theorem 2.1)
\thmnote{\ {\the\thm@notefont\sffamily\bfseries--- #3.}}} % Optional theorem note

\newtheoremstyle{nonum} % Theorem style name
{7pt} % Space above
{7pt} % Space below
{\normalfont} % Body font
{} % Indent amount
{\small\bf\sffamily} % Theorem head font
{\;\;} % Punctuation after theorem head
{0.25em} % Space after theorem head
{\small\sffamily\thmname{#1} % Theorem text (e.g. Theorem 2.1)
\thmnote{\ {\the\thm@notefont\sffamily\bfseries--- #3.}}} % Optional theorem note

\makeatother


% Defines the theorem text style for each type of theorem to one of the three styles above
\theoremstyle{chapcolornum}
%\newtheorem{theoremeT}{Theorem}[chapter]
\newtheorem{theoremeT}{Important}
\newtheorem{proposition}{Proposition}[chapter]
\newtheorem{problem}{Problem}[chapter]
\newtheorem{exerciseT}{Exercise}[chapter]
\theoremstyle{blacknumex}
\newtheorem{exampleT}{Example}[chapter]
\theoremstyle{blacknum}
\newtheorem{vocabulary}{Vocabulary}[chapter]
\newtheorem{definitionT}{Definition}[chapter]
\theoremstyle{nonum}
\newtheorem{corollaryT}{Learning objectives}



%----------------------------------------------------------------------------------------
%	DEFINITION OF COLORED BOXES
%----------------------------------------------------------------------------------------

\RequirePackage[framemethod=default]{mdframed} % Required for creating the theorem, definition, exercise and corollary boxes

% Theorem box
\newmdenv[skipabove=7pt,
skipbelow=7pt,
rightline=true,
leftline=true,
topline=true,
bottomline=true,
backgroundcolor=chapcolor!10,
linecolor=chapcolor,
innerleftmargin=5pt,
innerrightmargin=5pt,
innertopmargin=5pt,
innerbottommargin=5pt,
leftmargin=0cm,
rightmargin=0cm,
linewidth=1pt]{tBox}	



% Exercise box	  
\newmdenv[skipabove=7pt,
skipbelow=7pt,
rightline=false,
leftline=true,
topline=false,
bottomline=false,
backgroundcolor=chapcolor!10,
linecolor=chapcolor,
innerleftmargin=5pt,
innerrightmargin=5pt,
innertopmargin=5pt,
innerbottommargin=5pt,
leftmargin=0cm,
rightmargin=0cm,
linewidth=4pt]{eBox}	


% Definition box
\newmdenv[skipabove=10pt,
skipbelow=10pt,
rightline=false,
leftline=true,
topline=false,
bottomline=false,
linecolor=chapcolor,
innerleftmargin=5pt,
innerrightmargin=5pt,
innertopmargin=0pt,
leftmargin=0cm,
rightmargin=0cm,
linewidth=2pt,
innerbottommargin=0pt]{dBox}	

% Corollary box
\newmdenv[skipabove=7pt,
skipbelow=7pt,
rightline=false,
leftline=true,
topline=false,
bottomline=false,
linecolor=gray,
backgroundcolor=chapcolor!10,
innerleftmargin=5pt,
innerrightmargin=5pt,
innertopmargin=5pt,
leftmargin=0cm,
rightmargin=0cm,
linewidth=4pt,
innerbottommargin=5pt]{cBox}				  
		  

% Creates an environment for each type of theorem and assigns it a theorem text style from the "Theorem Styles" section above and a colored box from above
\newenvironment{theorem}{\begin{tBox}\begin{theoremeT}}{\end{theoremeT}\end{tBox}}
%\newenvironment{exercise}{\begin{eBox}\begin{exerciseT}}{\hfill{\color{chapcolor}\tiny\ensuremath{\blacksquare}}\end{exerciseT}\end{eBox}}				   %Removed black qed box in next line
\newenvironment{exercise}{\begin{eBox}\begin{exerciseT}}{\hfill{\color{chapcolor}}\end{exerciseT}\end{eBox}}
\newenvironment{definition}{\begin{dBox}\begin{definitionT}}{\end{definitionT}\end{dBox}}	
%\newenvironment{example}{\begin{exampleT}}{\hfill{\tiny\ensuremath{\blacksquare}}\end{exampleT}}		%Removed black qed box in next line
\newenvironment{example}{\begin{exampleT}}{\hfill{}\end{exampleT}}		
\newenvironment{corollary}{\begin{cBox}\begin{corollaryT}}{\end{corollaryT}\end{cBox}}	

%----------------------------------------------------------------------------------------
%	WARNING ENVIRONMENT
%----------------------------------------------------------------------------------------

\newenvironment{warning}{\par\vskip10pt\small % Vertical white space above the remark and smaller font size
\begin{list}{}{
\leftmargin=40pt % Indentation on the left
\rightmargin=25pt}\item\ignorespaces % Indentation on the right
\makebox[-2.5pt]{

 

\begin{tikzpicture}[overlay]
%\node[draw=chapcolor!60,line width=1pt,circle,fill=chapcolor!25,font=\sffamily\bfseries,inner sep=2pt,outer sep=0pt] at (-15pt,0pt){\textcolor{chapcolor}{R}};
  \node[anchor= north west] at (-40pt,15pt){\includegraphics[scale=0.45]{./Pictures/Icon_Stop/Icon-Stop.pdf} 
		}; %Semicolon important, separates icon from text
\end{tikzpicture}
} 
\advance\baselineskip -1pt}{\end{list}\vskip5pt} % Tighter line spacing and white space after remark

%----------------------------------------------------------------------------------------
%	IDEA ENVIRONMENT
%----------------------------------------------------------------------------------------

\newenvironment{idea}{\par\vskip10pt\small % Vertical white space above the remark and smaller font size
\begin{list}{}{
\leftmargin=35pt % Indentation on the left
\rightmargin=25pt}\item\ignorespaces % Indentation on the right
\makebox[-2.5pt]{

 

\begin{tikzpicture}[overlay]
%\node[draw=chapcolor!60,line width=1pt,circle,fill=chapcolor!25,font=\sffamily\bfseries,inner sep=2pt,outer sep=0pt] at (-15pt,0pt){\textcolor{chapcolor}{R}};
  \node[anchor= north] at (-15pt,0pt){\includegraphics[scale=1]{./Pictures/Icon_Idea/Idea-icon.pdf} 
		}; %Semicolon important, separates icon from text
\end{tikzpicture}
} 
\advance\baselineskip -1pt}{\end{list}\vskip5pt} % Tighter line spacing and white space after remark


%----------------------------------------------------------------------------------------
%	SECTION NUMBERING IN THE MARGIN
%----------------------------------------------------------------------------------------

\newcommand\numb[1]{{%
\begin{center}
\begin{tikzpicture}
\node (0,0) [rounded corners=25pt,fill=white,fill opacity=.6,text opacity=1,draw=chapcolor,draw opacity=1,line width=2pt,inner sep=15pt,align=
center]{\thesection \  \\ \makebox[15cm]{#1}};
\end{tikzpicture}
\end{center}
}}%

\titleformat{\section}[display]{\color{black}\normalfont\sffamily\bfseries\large}{}{0cm}{\numb}
\titlespacing{\section}{0pt}{0pt}{0pt}


\makeatletter
\renewcommand{\subsection}{\@startsection {subsection}{2}{\z@}
{-3ex \@plus -0.1ex \@minus -.4ex}
{0.5ex \@plus.2ex }
{\normalfont\sffamily\bfseries}}

\renewcommand{\subsubsection}{\@startsection {subsubsection}{3}{\z@}
{-2ex \@plus -0.1ex \@minus -.2ex}
{0.2ex \@plus.2ex }
{\normalfont\small\sffamily\bfseries}}                        
\renewcommand\paragraph{\@startsection{paragraph}{4}{\z@}
{-2ex \@plus-.2ex \@minus .2ex}
{0.1ex}
{\normalfont\small\sffamily\bfseries}}

%----------------------------------------------------------------------------------------
%	CHAPTER HEADINGS
%----------------------------------------------------------------------------------------

\newcommand{\thechapterimage}{}
\newcommand{\chapterimage}[1]{\renewcommand{\thechapterimage}{#1}}

\def\thechapter{\arabic{chapter}} 
\def\@makechapterhead#1{
\thispagestyle{empty}
{\centering \normalfont\sffamily
\ifnum \c@secnumdepth >\m@ne
\if@mainmatter
\startcontents
\begin{tikzpicture}[remember picture,overlay]
\node at (current page.north west)
{\begin{tikzpicture}[remember picture,overlay]

\node[anchor=north west,inner sep=0pt] at (0,0) {\includegraphics[width=\paperwidth]{\thechapterimage}};



\draw[anchor=west] (0cm,-12cm) node [rounded corners=25pt,fill=white,fill opacity=.6,text opacity=1,draw=chapcolor,draw opacity=1,line width=2pt,inner sep=15pt]{\huge\sffamily\bfseries\textcolor{black}{\thechapter\ ---\ #1\vphantom{plPQq} \makebox[15cm]{}}};


\end{tikzpicture}};
\end{tikzpicture}}
\par\vspace*{260\p@}
\fi
\fi
}
\def\@makeschapterhead#1{
\thispagestyle{empty}
{\centering \normalfont\sffamily
\ifnum \c@secnumdepth >\m@ne
\if@mainmatter
%\startcontents
\begin{tikzpicture}[remember picture,overlay]
\node at (current page.north west)
{\begin{tikzpicture}[remember picture,overlay]
\node[anchor=north west] at (-4pt,4pt) {\includegraphics[width=\paperwidth]{\thechapterimage}};
\draw[anchor=west] (3cm,-12cm) node [rounded corners=25pt,fill=white,opacity=.7,inner sep=15.5pt]{\huge\sffamily\bfseries\textcolor{black}{\vphantom{plPQq}\makebox[10cm]{}}};
\draw[anchor=west] (3cm,-12cm) node [rounded corners=25pt,draw=chapcolor,line width=2pt,inner sep=15pt]{\huge\sffamily\bfseries\textcolor{black}{#1\vphantom{plPQq}\makebox[10cm]{}}};
\end{tikzpicture}};
\end{tikzpicture}}
\par\vspace*{260\p@}
\fi
\fi
}
\makeatother





%%%%%%%%%%%%%%%% SECTION HEADINGS %%%%%%%%%%%%%%%%%%%%%
    %\titleformat{\section}{\LARGE\scseries}{\thesection}{1em}{}
		%\titleformat{\subsection}{\Large\bfseries}{\thesubsection}{1em}{}
		%\titleformat{\subsubsection}{\large\bfseries}{\thesubsubsection}{1em}{}
		%\titleformat{\paragraph}{\large\bfseries}{\theparagraph}{1em}{}




%%%%%%%       Zähler für Fragen und Aufgaben definieren      %%%%%%%%%%%
% 
% Die folgenden 'environments' stellen eine fortlaufende Z�hlung von 
% 'Fragen' und 'Aufgaben' �ber ein 'Xperiment' sicher; die Umgebungen k�nnen
% selbstverst�ndlich verlassen werden (um beliebigen anders formatierten 
% Text anzuf�hren).
% 
% F�r jeden Versuch soll ein K�rzel definiert werden, da� einer durchgehenden
% Z�hlung von 'Fragen' und 'Aufgaben' vorangestellt wird.
% 
%%%%%%                                                       %%%%%%%%%%%

\newcommand{\xpref}{XYZ}             % f"ur jeden Versuch Namen neu vergeben!!!

\newcounter{intZaehler}  \setcounter{intZaehler}{0}     % f"ur interne Z"ahlung
\newcounter{subZaehler}  \setcounter{subZaehler}{0}     % f"ur interne Z"ahlung
\newcounter{quZaehler}  \setcounter{quZaehler}{0}       % f"ur interne Z"ahlung

\newcounter{questZ}  
\setcounter{questZ}{0}        % f"ur jeden Versuch auf `0' setzen !!!

\newenvironment{question}{%                   produces an indented line
\begin{list}{\bfseries\xpref$\;$\ Q\, --\, %    within bold label and running text
    \arabic{quZaehler}}{%                     for a sequential numbering 
    \usecounter{quZaehler}%                   of questions throughout 
    \setcounter{quZaehler}{\value{questZ}}%   the entire section
    \setlength{\topsep}{0pt}%                  of an experiment;  
    \addtolength{\leftmargin}{12pt}%            form of label: 
    \addtolength{\labelsep}{5pt} }%            '>\xpref< Qu -- >extZaehler.<'
\setlength{\listparindent}{\parindent}%                   usage:
\setlength{\itemindent}{\parindent}%                      \begin{question}
\setlength{\parsep}{\parskip}\mdseries \slshape}%         \item  > text for question < 
{\end{list}\setcounter{questZ}{\value{quZaehler}}}%      \end{question}

\newcounter{taskZ} 
\setcounter{taskZ}{0}         % f"ur jeden Versuch auf `0' setzen !!!

\newenvironment{task}{%                           own environment to number 
\begin{list}{\bfseries\xpref$\;$\ T\, --\, %       `tasks'
        \arabic{intZaehler} }%                    within a section 
    {\usecounter{intZaehler}%                  (subsections may be labled
    \setcounter{intZaehler}{\value{taskZ}}%                 `Task -- x') 
    \setlength{\topsep}{0pt}% 
    \setlength{\leftmargin}{40pt}%
%    \setlength{\rightmargin}{10pt}%
    \addtolength{\labelsep}{5pt} }%
\setlength{\listparindent}{\parindent}%              usage:
\setlength{\itemindent}{\parindent}%                 \begin{task}
\setlength{\parsep}{\parskip}%                       \item > text for
\mdseries}%                                                    task < 
{\end{list}\setcounter{taskZ}{\value{intZaehler}}}%     \end{task}

\newcounter{subtaskZ}[taskZ] 
% automatically reset to `zero' when `taskZ' is incremented 
% !!! funktioniert nicht bei mir, 2005/11/30, uk
\setcounter{subtaskZ}{0}         % f"ur jeden neuen Task auf `0' setzen !!!

\newenvironment{subtask}{%                         nested environment to number 
\begin{list}{\bfseries %\ SubTask\ --\ %              `subtasks' sequentially 
        ( \alph{subZaehler} ) }%                   within a main task
    {\usecounter{subZaehler}%                    (subsections MAY be labled
    \setcounter{subZaehler}{\value{subtaskZ}}%            `SubTask -- x') 
    \setlength{\topsep}{0pt}%                  
%    \setlength{\rightmargin}{10pt}%
    \addtolength{\labelsep}{5pt} }%
\setlength{\listparindent}{\parindent}%            usage:
\setlength{\itemindent}{\parindent}%               \begin{subtask}
\setlength{\parsep}{\parskip}%                     \item  > text for
\mdseries}%                                                 subtask <
{\end{list}\setcounter{subtaskZ}{\value{subZaehler}}}%    \end{subtask}

%%Set paragraph format
\setlength{\parindent}{0cm}
\setlength{\parskip}{0mm}
